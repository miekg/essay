\documentclass[a4paper, 11pt]{article}

\usepackage{natbib}
\bibliographystyle{agsm}
\setcitestyle{authoryear,open={(},close={)}}

\usepackage{fontspec}
\setmainfont[Ligatures=TeX]{Lato Light}

\usepackage{graphicx}
\usepackage{wrapfig}

\linespread{1.05}

\makeatletter
\renewcommand{\maketitle}{
\begin{flushright}
{\LARGE\@title}

\vspace{50pt}

{\large\@author}
\\\@date 
\vspace{40pt}
\end{flushright}
}
\renewcommand{\@seccntformat}[1]{}
\makeatother


\title{\textbf{Essay}\\Sub title on Essay}

\author{\textsc{Ans Vaessen}
\\{\textit{ans@nlgids.london}}}

\date{\today}

%%%%%%%%%%%%%%%%% START %%%%%%%%%%%%%%%%%%

\begin{document} 
\maketitle 

\begin{abstract}
This is an abstract.
\end{abstract}

\vspace{30pt} % Some vertical space between the abstract and first section

\section*{Introduction}
Is Ryan air an disruptive entrant in the flight industry? And if so how have big firms like KLM and BA respond to this disruption.


Flying used to be an expensive mode of travel untill the low cost airlines entered the market. In 1995 two airlines came with a new concept: Low Cost Airlines. This was an innovation that changed the airline business as we know it. 




See \cite{Christensen97}.
See \citep{Christensen97}.
See \citep[p. 145]{Christensen97}.

Also see this website I found: \cite{TripAdvisor}.
:
\begin{quote}
hallo
\end{quote}

\section{Formatting Citations}

Ryanair and other low cost carriers started of came up in the mid nineties because of changing regulations. 
Flights used to be regulated by countries, each country had their own national airline. British Airways (BA) for Britain and (KLM) for the Netherlands for example. Governments would agree between nations on flying schedules, amount of passengers and fares. There was no competition until the European Commission introduced a reform proces that changed the rules. "Since April 1997, any airline holding a valid Air Operators Certificate in the EU can operate on any route within the European Union, including wholly within another country, without restriction on price or capacity. As a result, European air travel has been flooded with an influx of low-cost airlines \citep{Eurocontrol}. Low cost carriers like Ryanair made use of the changing regulation. This might also explain why it was difficult for airlines like KLM and BA to react to the changing market. They had more or less a monopoly and with these new regulations they were slow to react to the new Entrants. \citep{TiddBessant} Describe regulation as a source of innovation that works as a two sided sword, restricting on one side and deregulation "...may open up new innovative space. The liberalization and then privatization of telecommunications in many countries led to rapid growth in competition and high rates of innovation, for example" See \citep[p. 219]{TiddBessant}. 

Ryanair existed since 1984 but had already beginning of the nineties started with cutting the prices by copying the 'no frills' idea from Southwest airlines in the states. Calling themself the Europe’s first low fares airline \cite{Ryanair}. No food or meals and also moving to a single aircraft that keeps cost low in maintaince, inventory and training for pilots and offering the lowest fares. Helped by the new EU regulation they could expand and increase their flights to other countries. How was this disruptive?

\subsection{This is a subsection}
\label{sec:this-is-a-section}

\section{Another Section}


In his book The Innovater Dillema \cite{Christensen97} describes the innovator's dilemma and how incumbent fail. He describes how big companies sustain their growth by improving their products and performances. Often this is done in an incremental way, little steps at the time to fine tune and do better what they do good. It can be in a radical way as well. In his book he shows how the hard disk drive industry got disrupted multiple times. Could this also be the case for national airlines like BA and KLM are they disrupted by the so called low cost carriers like Ryanair?

According to Christensen \cite{Christensen97} there are 3 elements causing disruption:
Firstly, big companies want to better their performance. This can be done by innovation but has as goal to make a better product for the customer. Disruptive technology are often new entrants that look for a new fringe market to sell a inferior product often cheaper or simpler or both than the original product. Targeting a niche-market with less demanding customers. Secondly, the incumbents in their attempt to improve the product overshoot the mainstream market, whereas the new entrants have been improving their cheaper and simpler product and move into the mainstream market.
And thirdly, the main industry can't react to the disruption because it starts as a small market with low margins. To keep growing and making more profits they need higher margins. On top of that their customers demand a high quality product, they do not want a simpler product.

Graph 1.1 illustrates how this works. The big company is sustaining the technology and preformes better over time. So much so that they leave the mainstream and end up in the high end market. At the same time the new entrant is improving its product and enters the mainstream in time to take over from the incumbenmt \cite{Christensen97}.

graph Miek?


Let see how these elements hold up when applied to the airline business and Ryanair in particular.


\subsection{This is a subsection}
\label{sec:this-is-a-section}

\section{Another Section}


The Incumbents in this case are national airlines like BA and KLM. On their flights they provided food and unlimeted drinks, allocated seats, generous bagage allowence and overall good service. Flights leave from prime location and main easy to reach airports. Flights leave on prime times for a high price.They want to better their preformance. \cite{Christensen97} than contineous to describe how this can be disrupted by newcomers that offer a product of worse quality than the incumbent. Although in in the beginning not seen as a treath because of the inferieur design or performance it can sometimes turn out to be the cause of big firms failure. In this case study Ryanair started flying from smaller less fancy airports with lower landing charges. No allocated seats and non-reclining to add more pasageners, no free drinks or food and you pay for all the extra's. Their is no service. They will charge you for food, drinks, baggage, priority acces to the airplane etcetera. 
Instead of entering the regular market the Low Cost Airlines looked for new markets at the fringe. 

The main airlines kept doing what they do best and maybe even overshoot the mainstream market by adding more and more 'frills'. New users enter the market and started flying with Low cost carries like Ryanair and their market grew, these travellers talked about this 'good enough' flight and also the regular existing market including business travellers heard about it and took an interest in these 'cheaper' but good enough flights, especially on short haul flights \citep{TiddBessant}.

The mainstream carriers like BA and KLM faced the dilemma. Most of their customers still wanted the extra service. But in the end they gave in and started to make flights cheaper and ask people to pay for the extra's. Ryanair kept invating on the low cost base and the Internet was the next step.

Ryanair push through to the mainstream being innovative and launching a online bookingsite in 2000 reducing cost further no personel to take bookings or check in only baggage drop. 

\subsection{This is a subsection}
\label{sec:this-is-a-section}

\section{Another Section}


What did The big firms do 


Ryanair did not pose a threat at the start. But soon BA and KLM reacted with there own low cost GO and Buzz (find reference) 

Later quickly coppied things that worked.

But how can they stay mainstream ... KLM BA reacted and kept customers. 


Conclusion
The limits to disruption and how to make things cheaper are now entering the Human reaear departemnte and Bad publicity on how they treat personel make it unpopular there are many alternatives now.







\subsection{This is a subsection}
\label{sec:this-is-a-section}

\section{Another Section}


You can see in Section \ref{sec:this-is-a-section} on page \pageref{sec:this-is-a-section} that things are not going well here.

\bibliography{biblio}{}
\end{document}
